\documentclass{article}  
\usepackage[UTF8]{ctex}
\usepackage{minted}
\usepackage{geometry}
\usepackage{tcolorbox}
\usepackage{multicol}
\usepackage{xcolor}
\geometry{
    a4paper,
    % landscape,
    margin=1cm,
}
\begin{document}
\pagestyle{empty}
% \title{Cheatsheet for cheatsheet}
% \maketitle

\setlength{\columnsep}{30pt}
\begin{multicols}{2}

    \begin{minipage}{0.45\textwidth}
        \raggedright
        \begin{tcolorbox}[colback=orange!30, colframe=orange!60, title=Document]
            \begin{minted}[gobble=16]{tex}
                \documentclass{article} 
                \usepackage{geometry} 
                \geometry{
                    a4paper, landscape, margin=2cm,
                }
                % A4, Horizontal          

                \usepackage[UTF8]{ctex}
                % Chinese Charactor Support

                \pagestyle{empty} 
                % no page No.
            \end{minted}
        \end{tcolorbox}
        \begin{tcolorbox}[colback=cyan!30, colframe=cyan!60, title=Multi-columns]
            \begin{minted}[gobble=16]{tex}
                \usepackage{multicol}        
                \begin{multicols}{3}
                    % 3 cols, auto arrange, 
                    % or insert minipages  
                    \begin{minipage}{0.3\textwidth}  
                        % ...
                    \end{minipage}\hfill
                    % ...
                \end{multicols}                
            \end{minted}
        \end{tcolorbox}
    \end{minipage}\hfill
    \begin{minipage}{0.45\textwidth}
        \begin{tcolorbox}[colback=cyan!30, colframe=cyan!60, title=Colorful-text-box]
            \begin{minted}[gobble=16]{tex}
                \begin{tcolorbox}[colback=blue!30, 
                    colframe=blue!60, 
                    title=Title]
                    % content
                \end{tcolorbox}
        \end{minted}
        \end{tcolorbox}
        \begin{tcolorbox}[colback=violet!30, colframe=violet!60, title=Code-block]
            \begin{minted}[gobble=16]{tex}
                \begin{minted}[gobble=16]{tex}
                    % DO NOT support minited recursive
                    % or refer an input.tex  
                    % or use `\textbackslash`
                    % to replace `\`
                \end{ minted}  
                % here I put a white space 
            \end{minted}
        \end{tcolorbox}
        \begin{tcolorbox}[colback=orange!30, colframe=orange!60, title=Custom-color]
            \begin{minted}[gobble=16]{tex}
                \usepackage{xcolor}  
                \color{pink}  
                % or  
                \definecolor{mycolor}{RGB}{255,0,0}
                \definecolor{mycolor}{HTML}{FF0000}
                \definecolor{mycolor}{cmyk}{0,1,1,0}
                \color{mycolor}
            \end{minted}
        \end{tcolorbox}
        \begin{tcolorbox}[colback=magenta!30, colframe=magenta!60, title=Color]
            \begin{tabular}{llll}
                \textcolor{red}{red}         & \textcolor{green}{green}         & \textcolor{blue}{blue}   & \textcolor{cyan}{cyan}   \\
                \textcolor{magenta}{magenta} & \textcolor{yellow}{yellow}       & \textcolor{black}{black} & \textcolor{white}{white} \\
                \textcolor{orange}{orange}   & \textcolor{violet}{violet}       & \textcolor{brown}{brown} & \textcolor{pink}{pink}   \\
                \textcolor{purple}{purple}   & \textcolor{teal}{teal}           & \textcolor{olive}{olive} & \textcolor{lime}{lime}   \\
                \textcolor{gray}{gray}       & \textcolor{lightgray}{lightgray} &                          &                          
            \end{tabular}
        \end{tcolorbox}
        \begin{tcolorbox}[colback=cyan!30, colframe=cyan!60, title=.gitignore]
            \begin{tabular}{llll}
                *\_minted-* &  &  & \\
                *.log      &  &  & \\
                *.aux      &  &  & \\
                *.bbl      &  &  & \\
                *.synctex* &  &  &
            \end{tabular}
        \end{tcolorbox}

        \begin{tcolorbox}[colback=pink!30, colframe=pink!60, title=Others]
            1. add `--shell-escape` to VSCode `Latex Workshop`  \\
            2. how to solve `Underfull \textbackslash{hbox} (badness 10000)` warning.
        \end{tcolorbox}
    \end{minipage}
\end{multicols}

\end{document}